\documentclass{beamer}
%Od tukaj do polne vrstice % ne spreminjaj ničesar.
\usepackage[slovene]{babel}
\usepackage[utf8]{inputenc}
\usepackage{amsmath}
\usepackage{url}
\usepackage{graphicx}
\usepackage{tikz}
\usepackage{textpos}
\usepackage{wrapfig}

\definecolor{rdeca2017}{HTML}{8e0d1a}


\mode<presentation>
\usetheme{Boadilla}
\usefonttheme{structuresmallcapsserif}
\usenavigationsymbolstemplate{} 
\setbeamertemplate{footline}
{
  \leavevmode%
  \hbox{%
  \begin{beamercolorbox}[wd=.333333\paperwidth,ht=2.25ex,dp=1ex,center]{footline}%
    \usebeamerfont{author in head/foot}\setbeamercolor{bgcolor}{bg = rdeca2017}\insertshortauthor
  \end{beamercolorbox}%
  \begin{beamercolorbox}[wd=.333333\paperwidth,ht=2.25ex,dp=1ex,center]{footline}%
    \usebeamerfont{title in head/foot}\insertshorttitle
  \end{beamercolorbox}%
  \begin{beamercolorbox}[wd=.333333\paperwidth,ht=2.25ex,dp=1ex,right]{footline}%
    \insertframenumber{} / \inserttotalframenumber\hspace*{2ex} 
  \end{beamercolorbox}}%
  \vskip0pt%
}

\setbeamercolor{structure}{fg=rdeca2017}
\setbeamercolor{block title}{fg=white,bg=rdeca2017}
\setbeamercolor{block body}{fg=white,bg=rdeca2017}
\setbeamercolor{title}{bg = rdeca2017, fg = white}
\setbeamercolor{frametitle}{fg = white, bg = rdeca2017} 
\setbeamercolor{footline}{bg = rdeca2017, fg = white}

\addtobeamertemplate{frametitle}{}{%             %logo 
\begin{textblock*}{10mm}(.9\textwidth,-1cm)
\includegraphics[height=0.98cm]{logo_MaRS2017.png}
\end{textblock*}}

%%%%%%%%%%%%%%%%%%%%%%%%%%%%%%%%%%%%%%%%%%%%%%
% Tukaj lahko definiraš nova okolja ali oznake.
\theoremstyle{plain}
\newtheorem{izrek}{Izrek}
\newtheorem{trditev}{Trditev}
\newtheorem{lema}{Lema}
\newtheorem{definicija}{Definicija}
% Za številske množice uporabi naslednje simbole
\newcommand{\R}{\mathbb R}
\newcommand{\N}{\mathbb N}
\newcommand{\Z}{\mathbb Z}
\newcommand{\C}{\mathbb C}
\newcommand{\Q}{\mathbb Q}

%%%%%%%%%%%%%%%%%%%%%%%%%%%%%%%%%%%%%%%%%%%%%%%
%%%%%%%%%%%%%%%%%%%%%%%%%%%%%%%%%%%%%%%%%%%%%%%
% Izpolni naslov in avtorje s svojimi podatki
\title{KALEIDOCIKLI}
\author[Katharina, Katja, Tina, David]{\includegraphics[width = 5cm]{logo_MaRS2017.png} \\ Katharina Pavlin, Katja Kozlevčar, Tina Šafarič \\ mentor: David Gajser}
%\date{20.~avgust~2016}

\begin{document}
%\selectlanguage{slovene}


\begin{frame}
\maketitle

\end{frame}


\begin{frame}
\frametitle{Osnovni gradniki kaleidocikla}
%\begin{figure}[h!bt]
%\centering
%\includegraphics[width=15cm]{tetraeder1.png}
%\end{figure}
\end{frame}
%%%%%%%%%%%%%%%%%%%%%%%%%%%%%%%%%%%%%%%%%%%%%%%%%%%%%%%
\begin{frame}
\frametitle{Dve lastnosti naših tetraedrov}
%\begin{figure}[h!bt]
%\centering
%\includegraphics[width=12cm]{tetraeder.png}
%\end{figure}
\pause
\begin{enumerate}
\item $X$ je pravokoten na robova dolžine $a$.\pause
\item $|X| = \sqrt{b^2 - \frac{a^2}{2}}.$
\end{enumerate}

%\pause

\end{frame}


%%%%%%%%%%%%%%%%%%%%%%%%%%%%%%%%%%%%%%%%%%%%%%%%%%%%%%%%5


\begin{frame}
\frametitle{Osrednja formula za $2n$ tetraedrov}
\large

$$b\geq \frac{a}{2}\sqrt{2 + tan^2 (\frac{\pi}{n})}$$

\begin{itemize} \pause
\item $n=3$: \hspace {0.5cm} $\frac{b}{a}\geq \frac{\sqrt{5}}{2} \doteq 1.11803$ \pause
\item $n=4$: \hspace {0.5cm} $\frac{b}{a}\geq \frac{\sqrt{3}}{2} \doteq 0.86603$ \pause
\item $n=5$: \hspace {0.5cm} $\frac{b}{a}\geq \frac{1}{2}\sqrt{2 + tan^2 (\frac{\pi}{5})} \doteq 0.79496 $ \pause
\item $n\rightarrow \infty $: \hspace {0.195cm} $\frac{b}{a}\geq \frac{\sqrt{2}}{2} \doteq 0.70711 $ \pause
\end{itemize}

\end{frame}

\begin{frame}
\frametitle{Viri (ogled 16. 7. 2016)}
\begin{itemize}
\item Izdelava: \url{https://www.youtube.com/watch?v=kFUyXWlObLw}
\item Del teorije: \url{http://www.mathematische-basteleien.de/kaleidocycles.}
\item Patent (avtor Ralph M. Stalker): \url{http://www.google.com/patents/US1997022}
\item Računalniške simulacije: \url{http://intothecontinuum.tumblr.com/post/50873970770/an-even-number-of-at-least-8-regular-tetrahedra}
\end{itemize}
\end{frame}

\end{document}















