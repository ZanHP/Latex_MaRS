\documentclass[a4paper,12pt]{article}

\usepackage[slovene]{babel}
\usepackage{amsfonts,amssymb,amsmath}
\usepackage[utf8]{inputenc}
\usepackage[T1]{fontenc}
\usepackage{lmodern}
\usepackage{graphicx}
\usepackage[usenames, dvipsnames]{color}


\def\qed{$\hfill\Box$}   % konec dokaza
\def\qedm{\qquad\Box}   % konec dokaza v matematičnem načinu

\newtheorem{definicija}{Definicija}
\newtheorem{zgled}{Zgled}

\title{Grafovje in Številje}
\author{Neza Mujam}
\date{30.\ februar 2718}

%NAVODILA:

%Na računalniku bi sedaj poleg odprte datoteke moral/a imeti še tri slike: "eulerjeva", "petersenov_graf_barvanje" in "petersenov_graf_mnozice", pdf datoteko "lateh_delavnica", ki naj služi kot cilj tvojega preurejanja te datoteke in pa "lateh_cit_sit", ki ti bo zagotovo pomagal, ko malo osvojiš tovrstno pisanje.

%Da ti bo program barval kodo, pojdi na Oblika -> Barvanje kode -> LaTeX. Če želiš večjo pisavo pa Oblika -> Pisava ...

%Nad "\begin{document}" ti ni treba spremeniti ničesar.

%Začni na začetku, najboljše bo, če slediš besedilu, saj boš tako najlažje opazil/a, kje kaj manjka ali pa je treba besedilo oblikovati z ustreznimi ukazi. 



\begin{document}

%%%%
 
dodaj naslov

%%%%

Na Marsu

V teoriji grafov poznamo mnogo različnih tipov grafov. Razlikujemo jih glede na njihove specifične lastnosti. V tem seminarju se bomo ukvarjali s Kneserjevimi grafi oziroma podrobneje, s kromatičnim številom le-teh. Najprej podajmo nekaj glavnih definicij.

%%%%

DEFINICIJA
Graf G je urejen par (V,E), kjer je V množica vseh vozlišč, E pa množica vseh povezav med temi vozlišči.

DEFINICIJA
Graf ??, imenujemo ??, če je množica vozlišč ?? družina vseh $k$-elementnih podmnožic množice ??. Dve vozlišči sta povezani natanko takrat, ko sta disjunktni. 


Povejmo še, da za število vozlišč velja ??. V primeru, ko je ??, imata vsaki dve $k$-elementni množici neprazen presek. Tak Kneserjev graf nima nobenih povezav, zato privzemimo, da velja ??.

DEFINICIJA
Najmanjše število $m$, ki zadošča barvanju vozlišč grafa $G$, imenujemo ??. Označimo ga s ??.

DEFINICIJA
Preslikavo ??, ki slika vozlišča grafa v množico barv, imenujemo ??. Ta preslikava zadošča pogoju, da sta vsaki dve sosednji vozlišči pobarvani z različnima barvama.


Kromatično število grafa $G$ je torej najmanjše število barv, s katerimi lahko pobarvamo vozlišča grafa tako, da se nobeni dve sosednji vozlišči ne slikata v isto barvo. Množico vozlišč $V$ bi radi predstavili kot disjunktno unijo barvnih razredov ??, teh pa želimo, da je najmanj.



Oglejmo si enega najznamenitejših grafov vseh časov.

\begin{zgled}{Kneserjev graf $K(5,2)$ je zelo znan primer, imenujemo ga Petersenov graf. 

Dodaj sliki, najprej dodaj le eno sliko, če imaš preveč časa, pa se potrudi dodati še drugo.

Ta graf zelo pogosto pride prav pri dokazovanju obstoja nekega tipa grafa ali pa kot protiprimer. Imenuje se po \textbf{Juliusu Petersenu}, ki ga je skonstruiral za najmanjši kubični graf, to je, graf, ki je regularen stopnje $3$, brez mostov in brez barvanja povezav s tremi barvami. Ima $10$ vozlišč in $15$ povezav. Če je kdo besedilo bral, ga morda zanima, kaj pomeni, da je graf regularen. Ta podatek lahko zavzet bralec najde na \\ 
??.
}
\end{zgled}

\noindent
Dodaj rdečo črto.

\section{Po Marsu}
Po padcu z Marsa nas je ujel Euler. Kot naš rešitelj si je privoščil predstaviti nam svojo funkcijo iz teorije števil.

DEFINICIJA
Za vse ?? označimo število celih števil iz množice
??, ki so tuja številu $n$. Preslikavo ?? imenujemo Eulerjeva
funkcija.




\begin{table}[h!]
Tabela prikazuje izračun prvih šest vrednosti funkcije $\varphi (n)$. V $n$-ti
vrstici so krepko natisnjena števila med $1$ in $n$, ki so tuja številu $n$. Slika pa grafično prikazuje prvih $100$ vrednosti funkcije $\varphi (n)$.
\[
Dodaj tabelo, uporabi okolje "array". Če se ti ne da vpisovati vseh vrednost, prepiši le prvi dve vrstici.
\] 
Dodaj še napis pod tabelo, uporabi "caption".
\end{table}

SLIKA
Dodaj sliko "eulerjeva.PNG".

Zapišimo nekaj lastnosti te funkcije:

Potrudi se jih zapisati v okolje "itemize".

Dokažimo, da zadnja lastnost zagotovo velja za $n=6$.

{\emph Dokaz:}
V okolju "eqnarray" dokaži zadnjo lastnost za n=6.
\qed

Brez konteksta zapišimo še eno najlepših identitet, Eulerjevo seveda:
??Zapiši identiteto.


\end{document}